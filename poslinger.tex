\documentclass[a4paper]{article}
\usepackage[dutch]{babel} %duch document
\usepackage{amsmath} %math extra options
\usepackage{amsfonts} %math fonts
\usepackage{mathtools} %\abs command
\usepackage{titling}
\usepackage[hidelinks]{hyperref} %links in pdf
\usepackage[utf8]{inputenc} %correct typesetting
\usepackage{pdfpages} %import pdf pages
\usepackage{graphicx} %to include images
\usepackage{caption} %for figures without caption --> no :
\usepackage{subcaption} %for subfigure
\usepackage[section]{placeins} %figures before next section
\graphicspath{{images/}} %where images are stored
\newcommand\TestAppExists[3]{#2}
\usepackage{listings} %python

\DeclarePairedDelimiter\abs{\lvert}{\rvert}
\DeclarePairedDelimiter\norm{\lVert}{\rVert}

\newcommand{\subtitle}[1]{%
	\posttitle{%
		\par\end{center}
	\begin{center}\large#1\end{center}
	\vskip0.5em}%
	}

% Swap the definition of \abs* and \norm*, so that \abs and \norm resizes the size of the brackets, and the starred version does not.
\makeatletter
\let\oldabs\abs
\def\abs{\@ifstar{\oldabs}{\oldabs*}}

\let\oldnorm\norm
\def\norm{\@ifstar{\oldnorm}{\oldnorm*}}
\makeatother

\title{\textsc{De baan van een slinger met twee perioden}}

\subtitle{\textsc{Praktische opdracht}}

\author{\textsc{Vaughan Jones\thanks{Natuurkunde Tweede fase \textit{– Praktische opdracht –} Voorbereidend wetenschappelijk onderwijs (vwo), SSG Nehalennia Middelburg, 4335 AP, Nederland} \space en Jochem van Rabenswaaij\footnotemark[1]{}}}

\begin{document}
	\includepdf{voorkant.pdf}
	
	\maketitle
	\begin{abstract}
		\noindent Het bestuderen van bewegingen is erg belangrijk. In dit onderzoeksverslag worden de resultaten gepubliceerd van het onderzoek naar de beweging van een object dat slingert aan een tweedelige slinger die in twee verschillende richtingen slingert met in elke richting een eigen periode of trillingstijd. De reguliere manier van het beschrijven van deze beweging is door er van uit te gaan dat een slinger als harmonische trilling beschouwd mag worden. Voor grotere uitwijkingshoeken werkt dit niet. In dit verslag wordt stap voor stap uitgelegd hoe er een stelsel differentiaalvergelijkingen kan worden opgesteld die vervolgens gebruikt kunnen worden door met de methode van Euler een numeriek dynamisch model te programmeren. Vervolgens worden de proeven beschreven waarmee het model is getest en de resultaten van de metingen uiteengezet. Er zijn drie proeven uitgevoerd. De eerste proef was met een foto maken met een lange sluitertijd in een verduisterde kamer van een lampje aan een slinger met twee perioden. De tweede proef was doormiddel van een meetopstelling van hout en metalen verbindingsstukken met een telefoon waarbij de beweging werd geregistreerd met een gyroscoop. Gezien de meetonnauwkeurigheden bij deze twee methoden zijn we hebben we de laatste methode gekozen als meest nauwkeurige meetmethode: videometen. Doormiddel van een hoge resolutie camera zijn is er een beweging gefilmd die vervolgens handmatig met de videomeetsoftware CMA Coach 7 geanalyseerd is. De grote overeenstemming van het model en de resultaten wijzen erop dat het model niet verworpen hoeft te worden.
		\newline
		\newline
		\noindent \textbf{Trefwoorden:} \textit{Mathematische Slinger - Lissajous figuur - Dynamisch model - Numerieke benadering}
		
		
	\end{abstract}
	
	\clearpage
	\tableofcontents
	
	\clearpage
	\section{Inleiding}
	Het is erg belangrijk dat de wetenschap zich bezighoudt met het bestuderen van bewegingen. Al eeuwen kijken wetenschappers naar bewegende hemellichamen en ontdekt men universele regels over hoe de natuur zich gedraagt. Newton heeft een grote bijdrage geleverd aan het veralgemeniseren van de wetten waarmee de banen  kunnen worden beschreven die fysische objecten doorlopen wanneer ze een nettokracht ondervinden. Een van deze banen is de slingerbeweging. Een beweging die ontstaat omdat er een nettokracht gaat werken op een object omdat de krachten niet in evenwicht zijn. Welke baan doorloopt een slinger eigenlijk? Wanneer er twee slingers onder elkaar worden gehangen op zo een manier dat de ene slinger in een richting beweegt loodrecht op de andere slingerbeweging, is het dan nog steeds een normale slingerbeweging of is het een heel andere beweging geworden? Het is zeer belangrijk dat dit onderzocht wordt. Er moet onderzoek gedaan worden naar de wereld en de natuur om ons heen. Op academische niveau maar ook zeker op middelbare school niveau. Je bent namelijk pas echt met natuurkunde bezig als je de levenloze natuur zelf gaat bestuderen.
	Dit is een onderzoeksverslag van een onderzoek naar de baan van een slinger met twee perioden. Naar aanleiding van een YouTube video van Yeany (2017) \cite{Bruce} hebben wij besloten deze beweging te gaan bestuderen op verschillende manieren en om een dynamisch model te maken die deze beweging kan voorspellen.

	
	\section{Theorie}
	
	
	\subsection{Onderzoeksvraag}
	Welke baan doorloopt de puntmassa $P$ in de $\mathbb{R}^3$ waarbij punt $P$ het uiteinde is van een tweedelige slinger die in twee verschillende richtingen slingert, twee perioden of trillingstijden heeft en zich in een ruimte bevindt waar er zwaartekracht op het punt $P$ werkt?
	
	
	\subsection{Literatuuronderzoek}
	Een lissajousfiguur is een kromme die wordt gevormd door de baan van een punt dat gelijktijdig deelneemt aan twee onderling loodrechte harmonische trillingen. Volgens het artikel ``Lissajousfiguur''(z.d.)\cite{Lissajous} op wikipedia worden deze trillingen beschreven door:
	\[
	x = A \sin{(at+\delta)}
	\]
	en
	\[
	y = B \sin{(bt)}
	\]
	Deze vergelijkingen zijn gebaseerd op de wet van Hooke (``Wet van Hooke'', z.d.)\cite{Hooke}. Deze wet geldt voor de trilling van een veer, en bij benadering voor een slinger, wanneer de beginuitwijking heel klein genomen wordt. Dit betekent dat een slinger met twee perioden geen lissajous-figuur vormt.
	
	Wij vinden het belangrijk dat er een model wordt opgesteld dat de werkelijke baan van deze slinger beschrijft ongeacht de beginuitwijkingen. Aangezien wij op internet geen bestaande modellen gevonden hebben die voor alle beginhoeken werken, modelleren wij zelf een dynamisch model die numeriek de beweging kan benaderen met veel hogere nauwkeurigheid dan de reguliere methoden om slingerbewegingen te beschrijven.

	
	\subsection{Hypothese}
	We gaan uit van een $Oxyz$-assenstelsel. We noemen de oorsprong $O(0,0,0)$. Vanuit de oorsprong nemen we een punt $A$ dat op het $Oyz$-vlak ligt onder een hoek van $\alpha$ ten opzichte van de negatieve z-as. De vector die aangrijpt in de oorsprong en naar A gaat noemen we $ \vec{a}$ met de norm $\abs{\vec{a}}$. Door gebruik te maken van de goniometrische verhoudingen kan de vector uitgedrukt worden in $\abs{\vec{a}}$ en de hoek $\alpha$. De component in de $x$-richting is gelijk aan 0 en omdat de component in de $z$-richting naar beneden wijst wordt deze component negatief gemaakt. Zie figuur \ref{fig:vectora}.
	
	\begin{equation}
	\label{eq:vectora}
		\left.
		\begin{array}{rcl}
			\vec a &=& 	
				\begin{pmatrix}
					a_x \\
					a_y \\
					a_z
				\end{pmatrix} \\	
			a_x &=& 0 \\
			a_y &=& \abs{\vec{a}} \sin(\alpha) \\
			a_z &=& -\abs{\vec{a}} \cos(\alpha)
		\end{array} \right \}  \Rightarrow
		\vec{a} = 
		\begin{pmatrix}
			0 \\
			\abs{\vec{a}} \sin(\alpha) \\
			-\abs{\vec{a}} \cos(\alpha)
		\end{pmatrix}	
	\end{equation}
	
	\begin{figure}[htb]
		\centering
		\includegraphics[width=0.7\linewidth]{vectora}
		\caption{}
		\label{fig:vectora}
	\end{figure}
	
	We nemen vervolgens een punt $P$ dat op het vlak $PAO$ ligt dat evenwijdig is aan $\vec{a}$ en de $x$-as. De vector die aangrijpt in $A$ en naar $P$ gaat noemen we $\overrightarrow{AP}$ met de norm $\abs{\overrightarrow{AP}}$. De hoek tussen $\overrightarrow{AP}$ en $\vec{a}$ noemen we hoek $\beta$. De vector $\overrightarrow{AP}$ kan worden uitgedrukt in de hoeken $\alpha$ en $\beta$ en de norm $\abs{\overrightarrow{AP}}$ met goniometrische verhoudingen. De component in de $x$-richting is alleen afhankelijk van $\abs{\overrightarrow{AP}}$ en de hoek $\beta$ maar de componenten in de $y$- en de $z$-richting hangen daarnaast ook af van hoek $\alpha$. In het vlak $PAO$ is de $x$-component te berekenen en de afstand $d(A,P')$. Het punt $P'$ is de loodrechte projectie van punt $P$ op het $Oyz$-vlak. Met de afstand $d(A,P')$ zijn in het $Oyz$-vlak met behulp van F-hoeken de componenten in de $y$- en de $z$-richting te berekenen. Aangezien de component in de $z$-richting naar beneden wijst wordt deze component negatief gemaakt. Zie figuur \ref{fig:vectorAP1} en \ref{fig:vectorAP2}.
	
	\begin{equation}
	\label{eq:vectorAP}
		\left.
		\begin{array}{rcl}
			\overrightarrow{AP} &=& 
				\begin{pmatrix}
					AP_x \\
					AP_y \\
					AP_z 
				\end{pmatrix} \\
			AP_x &=& \abs{\overrightarrow{AP}} \sin(\beta) \\
			AP_y &=& d(A,P')\cdot\sin(\alpha) \\
			AP_z &=& d(A,P')\cdot\cos(\alpha) \\
			d(A,P') &=& \abs{\overrightarrow{AP}} \cos(\beta)
		\end{array} \right\} \Rightarrow
		\overrightarrow{AP} = 
			\begin{pmatrix}
				\abs{\overrightarrow{AP}}\sin(\beta) \\
				\abs{\overrightarrow{AP}}\sin(\alpha)\cos(\beta) \\
				-\abs{\overrightarrow{AP}}\cos(\alpha)\cos(\beta)
			\end{pmatrix}
	\end{equation}
	
	Door de vectoren $\vec{a}$ en $\overrightarrow{AP}$ op te tellen krijgen we de vector $\vec{p}$.
	
	\begin{equation}
	\label{eq:vectorp}
		\begin{array}{l}
			\vec{p} = \vec{a} + \overrightarrow{AP} =
				\begin{pmatrix}
					0 \\
					\abs{\vec{a}} \sin(\alpha) \\
					-\abs{\vec{a}} \cos(\alpha)
				\end{pmatrix} + 
				\begin{pmatrix}
					\abs{\overrightarrow{AP}}\sin(\beta) \\
					\abs{\overrightarrow{AP}}\sin(\alpha)\cos(\beta) \\
					-\abs{\overrightarrow{AP}}\cos(\alpha)\cos(\beta)
				\end{pmatrix} \\
			\vec{p} = 
				\begin{pmatrix}
					\abs{\overrightarrow{AP}}\sin(\beta) \\
					\abs{\overrightarrow{AP}}\sin(\alpha)\cos(\beta) + \abs{\vec{a}} \sin(\alpha) \\
					-\abs{\overrightarrow{AP}}\cos(\alpha)\cos(\beta) -\abs{\vec{a}} \cos(\alpha) 
				\end{pmatrix} \\
		\end{array}
	\end{equation}
	
	\begin{figure}[htb]
		\centering
		\includegraphics[width=0.6\linewidth]{vectorAP1}
		\caption{}
		\label{fig:vectorAP1}
	\end{figure}
	
	\begin{figure}[!h]
		\centering
		\includegraphics[width=0.9\linewidth]{vectorAP2}
		\caption{}
		\label{fig:vectorAP2}
	\end{figure}
	
	De vector $\vec{p}$ is de plaatsvector van een object dat slingert aan twee verbindingsstukken gerepresenteerd door de vectoren $\vec{a}$ en $\overrightarrow{AP}$. Op dit object werkt de zwaartekracht die voor een versnellingsvector evenwijdig aan de $z$-as zorgt dat naar beneden gaat. We noemen deze vector $\vec{g}$ met de norm $\abs{\vec{g}}$. De baanversnelling van punt $P$ in de slinger met een uitwijkingshoek van $\beta$ en straal $\abs{\overrightarrow{AP}}$ in het vlak $PAO$ noemen we $\abs{\overrightarrow{a_\beta}}$. Om $\abs{\overrightarrow{a_\beta}}$ uit te drukken in de hoeken $\alpha$ en $\beta$ en de valversnelling moet eerst de vector $\vec{g}$ ontbonden worden in componenten, waarvan er een in het $PAO$-vlak ligt en evenwijdig is aan de vector $\vec{a}$. Deze component noemen wij $\abs{\overrightarrow{g_\beta}}$. Met de goniometrische verhoudingen is deze component uit te drukken in $\alpha$ en $\abs{\vec{g}}$. Vervolgens kan $\abs{\overrightarrow{a_\beta}}$ uitgedrukt worden in $\beta$ en $\abs{\overrightarrow{g_\beta}}$. Dit is mogelijk omdat de baanversnelling loodrecht staat op de straal van de cirkelbeweging. Aangezien de versnelling tegen de bewegingsrichting in moet werken, wordt de baanversnelling negatief gemaakt, dit is ook wel bekend als de versnelling van een restauratiekracht, een kracht die naar de evenwichtsstand is gericht. Zie figuur \ref{fig:abeta}.
	
	\begin{equation}
	\label{eq:abeta}
		\left.
		\begin{array}{rcl}
			\abs{\overrightarrow{g_\beta}} &=& \abs{\vec{g}}\cos(\alpha) \\
			\abs{\overrightarrow{a_\beta}} &=& -\abs{\overrightarrow{g_\beta}}\sin(\beta)
		\end{array} \right\} \Rightarrow
		\abs{\overrightarrow{a_\beta}} = -\abs{\vec{g}}\cos(\alpha)\sin(\beta)
	\end{equation}
	
	\begin{figure}[htb]
		\centering
		\includegraphics[width=0.9\linewidth]{abeta}
		\caption{}
		\label{fig:abeta}
	\end{figure}
	
	Wanneer we in een cirkelbeweging een stuk cirkelboog afleggen leert de definitie van de radiaal ons dat de booglente gelijk is aan de hoek tussen het been door het middelpunt van de cirkel en door het startpunt op de cirkelboog, en het been door het middelpunt van de cirkel en door het eindpunt op de cirkelboog. Wanneer we de booglengte $s$ noemen, de hoek $\theta$ en de straal van de cirkel $r$ (zie figuur \ref{fig:circel}) kunnen we dit als volgt noteren:
	
	\begin{equation}
		s = \theta \cdot r
	\end{equation}
	
	\FloatBarrier
	\begin{figure}[htb]
		\centering
		\includegraphics[width=0.35\linewidth]{circel}
		\caption{}
		\label{fig:circel}
	\end{figure}
	
	\FloatBarrier
	We kijken nu naar de verandering in plaats per tijdseenheid. Hierdoor worden de booglente en de hoek functies van de tijd $t$ waarna we vervolgens beide leden van de vergelijking differentiëren naar $t$.
	
	\begin{equation}
		\begin{array}{rcl}
			s(t) &=& \theta(t) \cdot r \\
			\dfrac{ds(t)}{dt} &=& \dfrac{d\theta(t)}{dt} \cdot r
		\end{array}
	\end{equation}
	
	Om vervolgens de baanversnelling als een functie van de tijd te krijgen moet er nog een keer gedifferentieerd worden naar $t$.
	
	\begin{equation}
	\label{eq:diffs}
		\dfrac{d^2s(t)}{dt^2} = \dfrac{d^2\theta(t)}{dt^2} \cdot r
	\end{equation}
	
	In dit geval werken wij liever met de hoekversnelling. Wanneer we de tweede afgeleide van de functie $\theta(t)$ expliciet willen hebben uitgedrukt in de straal en de baanversnelling krijgen wij de volgende formule:
	
	\begin{equation}
	\label{eq:difftheta}
		\left.
		\begin{array}{rcl}
			\dfrac{d^2s(t)}{dt^2} &=& \dfrac{d^2\theta(t)}{dt^2} \cdot r \\
			a(t) &=& \dfrac{d^2s(t)}{dt^2}
		\end{array} \right\} \Rightarrow
		\dfrac{d^2\theta}{dt^2} (t) = \dfrac{a(t)}{r}
	\end{equation}
	
	Wanneer we in deze formule toepassen op de slinger met de uitwijkingshoek $\beta$, straal $\abs{\overrightarrow{AP}}$ en baanversnelling $\abs{\overrightarrow{a_\beta}}$ krijgen we een differentiaalvergelijking.
	
	\begin{equation}
	\label{eq:diffbeta}
		\left.
		\begin{array}{rcl}
			\dfrac{d^2\theta}{dt^2} (t) &=& \dfrac{a(t)}{r} \\
			\theta(t) &=& \beta(t) \\
			a(t) &=& \abs{\overrightarrow{a_\beta}} \\
			\abs{\overrightarrow{a_\beta}} &=& -\abs{\vec{g}}\cos(\alpha(t))\sin(\beta(t)) \\
			r &=& \abs{\overrightarrow{AP}}
		\end{array} \right\} \Rightarrow
		\dfrac{d^2\beta}{dt^2} (t) = -\abs{\vec{g}}\sin(\beta(t)) \dfrac{\cos(\alpha(t))}{\abs{\overrightarrow{AP}}}
	\end{equation}

	De kromme die het punt $P$ maakt in de $\mathbb{R} ^3$ als er zwaartekracht werkt op het punt $P$ hangt niet alleen af van $\beta(t)$ maar ook van $\alpha(t)$. Om differentiaalvergelijking op te stellen waarin de tweede afgeleide van $\alpha(t)$ voorkomt, moet een soortgelijke meetkundige situatie gemaakt worden als bij de slinger met de uitwijkingshoek $\beta(t)$. We stellen een vlak op zodat de gehele cirkelboog van de slingerbeweging zich daarin bevindt. Dit kan door een vlak op te stellen door de punten $P$, $B$ en $O$. Punt $B$ is het punt waar de vector $\vec{p}$ eindigt als $\alpha$ gelijk is aan nul. De vector $\vec{b}$ grijpt aan in de oorsprong $O$ en eindigt in het punt $B$. Vervolgens moet de valversnellingsvector $\vec{g}$ ontbonden worden in componenten, waarvan er een evenwijdig is aan het vlak $PBO$ en aan de vector $\vec{b}$. We noemen deze component $\abs{\overrightarrow{g_\alpha}}$. Door gebruik te maken van F-hoeken is aangetoond dat de hoek tussen $\vec{g}$ en $\overrightarrow{g_\alpha}$ gelijk is aan de hoek tussen de vectoren $\vec{b}$ en $\vec{c}$, waarbij $\vec{c}$ gelijk is aan $\vec{p}$ als $\alpha$ en $\beta$ gelijk zijn aan nul. Deze hoek noemen we hoek $\gamma$. $\abs{\overrightarrow{g_\alpha}}$ is met de goniometrische verhoudingen uit te drukken in de hoek $\gamma$ en de grootte van de valversnelling $\abs{\vec{g}}$. De baanversnelling van $P$ in de slingerbeweging met als evenwichtsstand vector $\vec{b}$ noemen we $\abs{\overrightarrow{a_\alpha}}$. Net als bij $\abs{\overrightarrow{a_\beta}}$ is deze uit te drukken in de valversnelling, de hoek waaronder het slingervlak zich bevindt ten opzichte van de valversnellingsvector en de grootte van de uitwijkingshoek. We noemen de uitwijkingshoek van de slinger met punt $B$ als evenwichtsstand hoek $\delta$. Dit is de hoek die gelijk is aan de hoek tussen de vectoren $\vec{p}$ en $\vec{b}$ wanneer deze groter dan of gelijk aan nul is. Waarom dit gesteld wordt zal nader worden verklaard. Wegens de gelijke hoeken $\angle(\vec{g},\overrightarrow{g_\alpha})$ en $\angle(\vec{b},\vec{c})$ kan $\abs{\overrightarrow{a_\alpha}}$ worden opgesteld, analoog aan de meetkundige situatie bij het uitdrukken van $\abs{\overrightarrow{a_\beta}}$ in $\abs{\vec{g}}$, $\alpha$ en $\beta$. Ook deze baanversnelling is een versnelling veroorzaakt door een restauratiekracht, en zal dus negatief gemaakt moeten worden. Zie figuur \ref{fig:aalpha}.
	
	\begin{equation}
	\label{eq:aalpha}
		\left.
			\begin{array}{rcl}
			\abs{\overrightarrow{g_\alpha}} &=& \abs{\vec{g}}\cos(\gamma) \\
			\abs{\overrightarrow{a_\alpha}} &=& -\abs{\overrightarrow{g_\alpha}}\sin(\delta)
		\end{array} \right\} \Rightarrow
		\abs{\overrightarrow{a_\alpha}} = -\abs{\vec{g}}\cos(\gamma)\sin(\delta)
	\end{equation}
	
	\begin{figure}[!h]
		\centering
		\includegraphics[width=0.9\linewidth]{aalpha}
		\caption{}
		\label{fig:aalpha}
	\end{figure}

	Door gebruik te maken van de stelling dat de tweede afgeleide van een hoek als functie van de tijd gelijk is aan de baanversnelling van dat punt als functie van de tijd gedeeld door de straal van de cirkel die het punt doorloopt, kan er van de baanversnelling $\abs{\overrightarrow{a_\alpha}}$ een hoekversnelling gemaakt worden.
	
	\begin{equation}
	\label{eq:diffalpha}
		\left.
		\begin{array}{rcl}
			\dfrac{d^2\theta}{dt^2} (t) &=& \dfrac{a(t)}{r} \\
			\theta(t) &=& \beta(t) \\
			a(t) &=& \abs{\overrightarrow{a_\alpha}} \\
			\abs{\overrightarrow{a_\alpha}} &=& -\abs{\vec{g}}\cos(\gamma(t))\sin(\delta(t)) \\
			r &=& \abs{\vec{p}}
		\end{array} \right\} \Rightarrow
		\dfrac{d^2\alpha}{dt^2} (t) = -\abs{\vec{g}}\sin(\delta(t)) \dfrac{\cos(\gamma(t))}{\abs{\vec{p}}}
	\end{equation}
	
	We kunnen de differentiaalvergelijking hierboven door middel van substitutie uitdrukken in de tweede afgeleide van de hoek als functie van de tijd, de valversnelling, de lengtes van vectoren $\vec{a}$ en $\overrightarrow{AP}$ en de hoeken $\alpha$ en $\beta$. $\cos(\gamma)$ kan vervangen worden door een uitdrukking met $\beta$, $\abs{\overrightarrow{AP}}$ en $\abs{\vec{a}}$. Dit kan door de hoek uit te rekenen tussen de lijnen met de richtingsvectoren $\vec{c}$ en $\vec{b}4$. Hierbij is $\vec{c}$ de vector die aangrijpt in $O$ en naar punt $C$ gaat. Punt $C$ valt samen met $P$ wanneer $\alpha$ en $\beta$ gelijk zijn aan nul. De cosinus van een hoek tussen twee vectoren is uit te drukken in het inwendig product van de desbetreffende vectoren en de bijbehorende normen van deze vectoren. Dit verband is afgeleid van de cosinusregel.

	\begin{equation}
		\cos(\gamma) = \dfrac{\vec{b} \cdot \vec{c}}{\abs{\vec{b}} \cdot \abs{\vec{c}}}
	\end{equation}
	
	De vector $\vec{b}$ is afhankelijk van de lengtes $\abs{\overrightarrow{AP}}$ en $\abs{\vec{a}}$ en de hoek $\beta$ en de vector $\vec{c}$ hangt alleen af van de lengtes $\abs{\overrightarrow{AP}}$ en $\abs{\vec{a}}$. Hiermee is ons doel van het vervangen van $\cos(\gamma)$ bereikt. De andere factor uit de differentiaalvergelijking van de tweede afgeleide van hoek $\alpha$ als een functie van de tijd die nog vervangen moet worden is $\sin(\delta)$. We kunnen dit uitdrukken in de hoeken $\alpha$ en $\beta$ en de lengtes $\abs{\overrightarrow{AP}}$ en $\abs{\vec{a}}$ door gebruik te maken van het uitwendig product bij de vectoren $\vec{p}$ en $\vec{b}$. Hierbij maken we gebruik van de regel die zegt dat de norm van het uitwendig product van twee vectoren in een driedimensionale ruimte gelijk is aan het oppervlakte van het parallellogram dat wordt ingesloten door de vectoren wanneer de parallellogrammethode wordt toegepast bij een sommatie van twee vectoren. De oppervlakte van een parallellogram $O_{parallellogram}$ is gelijk aan de basis maal de hoogte $h$ van dat parallellogram. De basis is in dit geval gelijk aan de norm van één van de twee vectoren, wij kiezen hier voor $\vec{b}$ als basis. Met behulp van de goniometrische verhoudingen is vervolgens de hoogte het parallellogram uit te drukken in de lengte van de zijde die schuin is ten opzichte van de basis. De schuine zijde is in deze situatie $\abs{\vec{p}}$. Wanneer we deze verbanden samenvoegen komen we op een vergelijking waarna we vervolgens $\sin(\delta)$ expliciet kunnen schrijven.
	
	\begin{equation}
	\label{eq:sindelta}
		\left.
		\begin{array}{rcl}
			O_{parallellogram} &=& \abs{\vec{p} \cdot \vec{b}} \\
			O_{parallellogram} &=& \abs{\vec{b}} \cdot h \\
			h &=& \abs{\vec{p}} \cdot \sin(\delta)
		\end{array} \right\} \Rightarrow
	\end{equation}
	\[
	\Rightarrow \abs{\vec{p} \cdot \vec{b}} = \abs{\vec{b}} \abs{\vec{p}} \cdot \sin(\delta)
	\Rightarrow \sin(\delta) = \dfrac{\abs{\vec{p} \cdot \vec{b}}}{\abs{\vec{b}} \abs{\vec{p}}}
	\]
	
	Zoals eerder ter sprake is gekomen, is $\delta$ gelijk aan de hoek tussen de vectoren $\vec{p}$ en $\vec{b}$ wanneer deze groter dan of gelijk aan nul is. Dit komt doordat de normen allemaal groter dan of gelijk aan nul zijn in het rechterlid in de vergelijking hierboven. Maar dit levert problemen op. De hoek $\delta$ is de uitwijkingshoek van de slinger die als evenwichtsstand punt $B$ heeft. Zoals het momenteel gesteld is kan de sinus van hoek $\delta$ geen negatieve waarden aannemen. Wanneer we gebruik maken van de eenheidscirkel zien wij dat $\delta$ hierdoor niet negatief kan worden. De factor $\sin(\delta)$ in de differentiaalvergelijking een rol zou moeten hebben met het bepalen van de richting. De richting van de hoekversnelling is namelijk afhankelijk van de uitwijkingshoek. Er moet dus een richtingbepalende factor worden toegevoegd aan de differentiaalvergelijking. De hoekversnelling moet positief zijn wanneer de uitwijkingshoek negatief is en de hoekversnelling moet negatief zijn wanneer de uitwijkingshoek positief is. De grens waarop de omslag van positief naar negatief of andersom plaatsvindt is wanneer de uitwijkingshoek gelijk is aan nul en wanneer de uitwijkingshoek gelijk is aan veelvouden van $\pi$. Hiervoor kan de hoek $\alpha$ gebruikt worden. De hoek $\alpha$ is gelijktijdig met de uitwijkingshoek positief en negatief. Een oplossing voor de omslagpunten van negatief naar positief of andersom is het gebruiken van de sinus van hoek $\alpha$. We willen slechts de hoekversnelling vermenigvuldigen met -1 afhankelijk van de hoek $\alpha$. Hiervoor vermenigvuldigen we de hoekversnelling met $\sin(\alpha)$ gedeeld door de modulus van $\sin(\alpha)$. We noemen deze factor de richtingsbepalende factor $f(\alpha)$. In figuur \ref{fig:f(x)} is te zien wat voor grafiek dat oplevert.
	
	\begin{equation}
	\label{f(x)}
		f(\alpha) = \dfrac{\sin(\alpha)}{\abs{\sin(\alpha)}}
	\end{equation}
	
	\begin{figure}[htb]
		\centering
		\includegraphics[width=\linewidth]{f(x)}
		\caption{}
		\label{fig:f(x)}
	\end{figure}

	Wanneer we $f(\alpha)$, $\sin(\delta)$ en $\cos(\gamma)$ substitueren in de differentiaalvergelijking met de tweede afgeleide van $\alpha$ als functie van de tijd geeft dat het volgende:
	
	\begin{equation}
	\label{eq:diffalpha2}
		\left.
		\begin{array}{rcl}
			\dfrac{d^2\alpha}{dt^2} (t) &=& -f(\alpha(t)) \abs{\vec{g}} \sin(\delta(t)) \dfrac{\cos(\gamma(t))}{\abs{\vec{p}}} \\
			\cos(\gamma) &=& \dfrac{\vec{b} \cdot \vec{c}}{\abs{\vec{b}} \cdot \abs{\vec{c}}} \\
			\sin(\delta) &=& \dfrac{\abs{\vec{p} \times \vec{b}}}{\abs{\vec{b}} \abs{\vec{p}}} \\
			f(\alpha(t)) &=& \dfrac{\sin(\alpha(t))}{\abs{\sin(\alpha(t))}}
		\end{array} \right\} \Rightarrow
	\end{equation}
	\[
	\Rightarrow \dfrac{d^2\alpha}{dt^2} (t) = -\abs{\vec{g}} \cdot \dfrac{\sin(\alpha(t))}{\abs{\sin(\alpha(t))}} \cdot \dfrac{\abs{\vec{p} \times \vec{b}}}{\abs{\vec{b}} \abs{\vec{p}}} \cdot \dfrac{\vec{b} \cdot \vec{c}}{\abs{\vec{b}} \cdot \abs{\vec{c}}} \cdot \dfrac{1}{\abs{\vec{p}}}
	\]
	
	De normen van de vectoren $\vec{p}$ en $\vec{b}$ aan elkaar gelijk zijn, de vector $\vec{b}$ gelijk is aan de vector $\vec{p}$ als $\alpha$ gelijk aan nul is en de vector $\vec{c}$ gelijk is aan de vector $\vec{p}$ als $\alpha$ en $\beta$ gelijk zijn aan nul. Hierdoor kan de differentiaalvergelijking als volgt genoteerd worden:
	
	\begin{equation}
		\dfrac{d^2\alpha}{dt^2} (t) = -\abs{\vec{g}} \sin(\alpha(t)) \dfrac{\abs{\vec{p} \times \vec{b}}(\vec{b} \cdot \vec{c})} {\abs{\sin(\alpha(t))}\abs{\vec{c}} \abs{\vec{b}}^4}
	\end{equation}
	
	Het stelsel differentiaalvergelijkingen van de hoekversnellingen van de hoeken $\alpha(t)$ en $\beta(t)$ zijn in de volgende matrix gezet en dit stelsel is onze hypothese. Wanneer de oplossingen van dit stelsel een parametervoorstelling wordt van hoek $\alpha$ en $\beta$ als functie van de tijd en vervolgens een parameterkromme wordt getekend in de $\mathbb{R}^3$ van de baan van punt $P$, dan zal hier de baan van een slinger met twee perioden worden getekend met 2 verbindingsstukken waarbij zwaartekracht werkt op een puntmassa onderaan de slinger.
	
	\begin{equation}
		\dfrac{d^2}{dt^2} 
		\begin{pmatrix}
			\alpha(t) \\
			\beta(t)
		\end{pmatrix} =
		 -\abs{\vec{g}} 
		 \begin{pmatrix}
			\sin(\alpha(t)) & 0\\
			0 & \sin(\beta(t)) 
		\end{pmatrix}
		\begin{pmatrix}
			 \dfrac{\abs{\vec{p} \times \vec{b}}(\vec{b} \cdot \vec{c})} {\abs{\sin(\alpha(t))}\abs{\vec{c}} \abs{\vec{b}}^4} \\
			 \dfrac{\cos(\alpha(t))}{\abs{\overrightarrow{AP}}}
		\end{pmatrix}
	\end{equation}
	
	Met:
	
	\[
	\vec{p} = 
	\begin{pmatrix}
		\abs{\overrightarrow{AP}}\sin(\beta) \\
		\abs{\overrightarrow{AP}}\sin(\alpha)\cos(\beta) + \abs{\vec{a}} \sin(\alpha) \\
		-\abs{\overrightarrow{AP}}\cos(\alpha)\cos(\beta) -\abs{\vec{a}} \cos(\alpha) 
	\end{pmatrix}, 
	\vec{b} = 
	\begin{pmatrix}
		\abs{\overrightarrow{AP}}\sin(\beta) \\
		0 \\
		-\abs{\overrightarrow{AP}}\cos(\beta) -\abs{\vec{a}}
	\end{pmatrix},
	\]
	\[
	\vec{c} = 
	\begin{pmatrix}
		0 \\ 0 \\
		-\abs{\overrightarrow{AP}} -\abs{\vec{a}}
	\end{pmatrix},
	\abs{\vec{g}} = 9,81
	\]
	
	Door gebruik te maken van de methode van Euler kan er een numerieke benadering gemaakt worden van de oplossingen van het stelsel differentiaalvergelijkingen hierboven. Bij deze methode wordt er een differentiaalquotiënt omgeschreven naar een vermenigvuldiging. Stel een functie $f(x)$ voor. De afgeleide van deze functie kan door een vermenigvuldiging omgeschreven worden naar een differentiaal:
	
	\begin{equation}
		\begin{array}{rcl}
			\dfrac{d}{dx}f(x) &=& \dfrac{f(x+dx) - f(x)}{dx} \\ \\
			\dfrac{d}{dx}f(x)dx &=& \dfrac{f(x+dx) - f(x)}{dx}dx = f(x+dx) - f(x) = df(x)
		\end{array}
	\end{equation}
	
	Omschrijven van $f(x+dx) - f(x) = df(x)$ geeft:
	
	\begin{equation}
		f(x+dx) = f(x) + df(x)
	\end{equation}
	
	Maar we weten dat $df(x) = \frac{d}{dx} f(x)dx=f'(x)dx$, substitutie hiervan in de bovenstaande vergelijking geeft het volgende:
	
	\begin{equation}
		f(x+dx) = f(x) + f'(x)dx
	\end{equation}
	
	Bij de methode van Euler wordt er dus een kleine toename in functiewaarde berekent die vervolgens wordt opgeteld bij de oude functiewaarde. Dit resulteert in een nieuwe functiewaarde. Vervolgens wordt $x$ verhoogt met $dx$ en wordt de berekening opnieuw uitgevoerd. Door bij modelleersoftware $dx$ maar klein genoeg te kiezen kan met behulp van numerieke modellen een goede benadering verkregen worden van het verloop van de functie. Dit is wat wij gebruiken om ons model de baan van punt $P$ in de $\mathbb{R}^3$ te laten berekenen. Het model is te vinden in bijlage B. 
	
	
	
	\section{Methode en resultaten}
	
	We hebben op verschillende manieren gemeten, omdat de meetmethodes voortdurend niet nauwkeurig genoeg bleken te zijn. Op basis van de onnauwkeurigheden van de ene methode werd steeds een nieuwe methode bedacht. Er zijn uiteindelijk drie methoden gebruikt.
	
	Bij methode 2 en 3 hebben we de resultaten geplot met de programmeertaal Python 3 (zie bijlage B). We hebben het model ingevoerd in Python en de meetgegevens via een csv-bestand geïnporteerd. Door die samen in één figuur te plotten kunnen model en werkelijkheid met elkaar vergeleken worden. Op deze manier zijn alle figuren uit dit verslag geplot. 
	
	\subsection{Methode 1}
	\label{met1}	
	De eerste meting (zie figuur \ref{fig:sluiter}) is gedaan in een donkere ruimte met behulp van een ledlampje opgehangen aan een dubbele slinger gemaakt van touw. Hieronder ligt een camera met de lens naar boven, en met deze camera wordt een foto gemaakt met een lange sluitertijd. Hierdoor wordt de totale afgelegde weg van het lampje in beeld gebracht.
	
	\subsubsection*{Materialen}
	\begin{itemize}
		\setlength\itemsep{0em}
		\item 4m touw
		\item stok
		\item ledlampje met batterij
		\item fotocamera
	\end{itemize}
	
	\subsubsection*{Resultaten}
	In figuur \ref{fig:sluiter} is een foto te zien gemaakt volgens deze methode: een camera onder een slinger, waaraan een lampje hangt.
	
	\begin{figure}[!h]
		\centering
		\includegraphics[width=0.5\linewidth]{sluitertijd}
		\caption{Resultaat methode 1}
		\label{fig:sluiter}
	\end{figure}
	\FloatBarrier
	
	\subsection{Methode 2}
	\label{met2}
	Vervolgens wordt er gemeten met sensoren van een telefoon. Hiervoor moet eerst een andere opstelling worden gebouwd omdat de telefoon niet mag draaien tijdens de slingerbeweging, wat aan een opstelling gemaakt van touw bijna onvermijdelijk is. Daarom wordt er een opstelling gebouwd met metalen staven als slingerarmen (zie figuur \ref{fig:opst-telefoon}). Hieronder wordt een telefoon gemonteerd.
	\begin{figure}[htb]
		\centering
		\includegraphics[width=0.5\linewidth]{opst-telefoon}
		\caption{De meetopstelling van methode 2}
		\label{fig:opst-telefoon}
	\end{figure}
	Op de telefoon wordt een applicatie geïnstalleerd die de meetgegevens van de sensoren exporteert naar een csv-bestand. Als eerste wordt dit gedaan met een gravitatiesensor. Dit blijkt een zeer grof resultaat te geven, dus wordt besloten om een meting uit te voeren met een gyroscoop. Deze metingen blijken veel nauwkeuriger. Door de beginuitwijking en de lengte van beide delen van de slinger te meten, wordt via het model een grafiek gegenereerd voor dezelfde omstandigheden. Door in één figuur de meetgegevens en de gegevens verkregen met het model te plotten, worden de resultaten vergeleken.
	
	\subsubsection*{Materialen}
	\begin{itemize}
		\setlength\itemsep{0em}
		\item 6 houten latjes
		\item 3 metalen staven
		\item 4 houten blokjes
		\item selfiestick
		\item metalen buigbaar staafje met gaatjes
		\item moertjes, boutjes, ringetjes, spijkers en schroeven naar believen
		\item telefoon met gyrosensor
		\item Python 3 software
	\end{itemize}

	\subsubsection*{Resultaten}
	
	\begin{figure}[!h]
		\centering
		\includegraphics[width=0.8\linewidth]{telefoon(1)}
		\caption{Resultaat van methode 2: de uitwijking in de $y$-richting uitgezet tegen de uitwijking in de $x$-richting}
		\label{fig:telefoon(1)}
	\end{figure}
	
	\begin{figure}[!h]
		\centering
		\includegraphics[width=0.8\linewidth]{telefoon(2)}
		\caption{Resultaat van methode 2: de uitwijking in de $y$-richting uitgezet tegen de tijd}
		\label{fig:telefoon(2)}
	\end{figure}
	
	\begin{figure}[!h]
		\centering
		\includegraphics[width=0.8\linewidth]{telefoon(3)}
		\caption{Resultaat van methode 2: de uitwijking in de $x$-richting uitgezet tegen de tijd}
		\label{fig:telefoon(3)}
	\end{figure}
	\FloatBarrier
	
	\subsection{Methode 3}
	\label{met3}
	De volgende manier van meten is met behulp van de analyse van videobeelden van een slingerbeweging. Een dubbele slinger gemaakt van touw hangt aan een statief. Op het statief is een telefoon bevestigd met een selfiestick (zie figuur \ref{fig:opst-video}). De telefoon filmt in 4K-resolutie.
	
	\begin{figure}[htb]
		\centering
		\begin{subfigure}{.5\textwidth}
			\centering
			\includegraphics[angle=-90,origin=c,width=0.6\linewidth]{opst-video(1)}
			\caption{De slinger bevestigd aan het statief}
			\label{fig:opst-video(1)}
		\end{subfigure}%
		\begin{subfigure}{.5\textwidth}
			\centering
			\includegraphics[angle=-90,origin=c,width=0.6\linewidth]{opst-video(2)}
			\caption{De telefoon bevestigd op het statief}
			\label{fig:opst-video(2)}
		\end{subfigure}
		\caption{De meetopstelling van methode 3}
		\label{fig:opst-video}
	\end{figure}

	Er wordt een video gemaakt van een slingerbeweging van ongeveer 10 seconden. De video wordt in Coach 7 (http://cma-science.nl) videometen ingeladen, waarna handmatig in elk beeldje (60 fps) de positie van het gewichtje wordt gemarkeerd (zie figuur \ref{fig:verw-video}). De hiermee verkregen data worden met behulp van Python 3 in een grafiek gezet. 
	
	\begin{figure}[!h]
		\centering
		\includegraphics[width=\linewidth]{verw-video}
		\caption{Verwerking van de videobeelden met Coach 7}
		\label{fig:verw-video}
	\end{figure}
	
	Door de beginuitwijking en de lengte van beide delen van de slinger te meten, wordt via het model een grafiek gegenereerd voor dezelfde omstandigheden. Door beide grafieken in dezelfde figuur te plotten, worden de resultaten vergeleken.
	
	\subsubsection*{Materialen}
	\begin{itemize}
		\setlength\itemsep{0em}
		\item 4m touw
		\item statief
		\item gewichtje
		\item telefoon met 4K opnamefunctie
		\item selfiestick met mogelijkheid tot bevestiging aan het statief
		\item Coach 7 en Pyton 3 software
	\end{itemize}
	
	\subsubsection*{Resultaten}
	\begin{figure}[!h]
		\centering
		\includegraphics[width=0.8\linewidth]{Videometen(1)}
		\caption{Resultaat van methode 3: de uitwijking in de $y$-richting uitgezet tegen de uitwijking in de $x$-richting}
		\label{fig:Videometen(1)}
	\end{figure}
	
	\begin{figure}[!h]
		\centering
		\includegraphics[width=0.8\linewidth]{Videometen(2)}
		\caption{Resultaat van methode 3: de uitwijking in de $y$-richting uitgezet tegen de tijd}
		\label{fig:Videometen(2)}
	\end{figure}
	
	\begin{figure}[!h]
		\centering
		\includegraphics[width=0.8\linewidth]{Videometen(3)}
		\caption{Resultaat van methode 3: de uitwijking in de $x$-richting uitgezet tegen de tijd}
		\label{fig:Videometen(3)}
	\end{figure}
	
	
	
	\section{Discussie en conclusie}
	In dit hoofdstuk wordt de bruikbaarheid van de drie achtereenvolgens gebruikte methoden besproken. Daarna wordt in de paragraaf 'Conclusie' de hypothese getoetst aan de hand van de resultaten van de meest nauwkeurige meetmethode.
	
	\subsection{Discussie methoden}
	Bij methode 1 ligt de camera door de beperkte ruimte vlak onder het lampje dat de slingerbeweging maakt. Het beeld wordt hierdoor door de lens vervormd. Bovendien is niet terug te zien waar het lampje zich bevindt op een bepaald tijdstip en is de schaal van de foto erg lastig te bepalen, doordat het geheel zich in een donkere kamer moet afspelen.
	
	Bij methode 2 kunnen de resultaten niet goed worden gemeten, omdat de massa van de slinger zo groot is dat het zwaartepunt van de slinger significant hoger ligt dan de telefoon. Om dit op te lossen zouden de beide lengtes met een correctiefactor moeten worden verkort, wat de nauwkeurigheid van de metingen sterk vermindert. Bovendien heeft de slinger veel wrijving, wat het vergelijken met een model waarin geen wrijving is meegenomen ingewikkeld maakt. Ook blijken de sensoren van de telefoon onnaukeurig, omdat ze zijn gemaakt om bijvoorbeeld mee te gamen, niet om practica mee uit te voeren.
	
	Methode 3 blijkt de meest nauwkeurige methode. Door de camera hoog boven de opstelling te plaatsen, wordt de vervorming geminimaliseerd (verbetering ten opzichte van methode 1). Door de slinger te maken van touw is de massa van de slinger geminimaliseerd, waardoor het zwaartepunt ongeveer op het zwaartepunt van de massa onder de slinger ligt. Bovendien is de wrijving van deze opstelling verwaarloosbaar. Doordat op elk beeld de positie van het gewichtje digitaal is gemarkeerd, zijn sensoronnauwkeurigheiden uitgesloten (verbeteringen ten opzichte van methode 2).
	
	\subsection{Conclusie}
	In dit onderzoek is gezocht naar een antwoord op de vraag: 'Welke baan beschrijft punt $P$ in de $\mathbb{R}^{3}$ wanneer daar alleen zwaartekracht op werkt, met het punt $P$ het onderste punt van een slinger met twee perioden.' Het antwoord op deze vraag is gezocht door het maken van een wiskundig model en het meten van een fysieke slinger in een meetopstelling. 
	
	Aangezien de resultaten van beide onderdelen dicht in de buurt van elkaar liggen, denken wij te mogen concluderen dat wij een model hebben opgesteld dat de werkelijke baan van een slinger met twee perioden dicht benadert.
	
	\subsection{Foutendiscussie}
	De nauwkeurigheid van de meetmethode is nog voor verbetering vatbaar, omdat doordat we gebruik maken van een camera, zal het beeld nog wel een beetje vervormd worden, en bij het videometen moesten de punten $P$ met de hand worden aangeklikt, waardoor daar een flinke onnaukeurigheid in zit.  Voor vervolgonderzoek is aan te bevelen een goede kwaliteit camera te gebruiken, en het aanklikken van de meetpunten te automatiseren.
	
	De bepaling van de startpositie hebben wij gedaan met goniometrische verhoudingen en bekende lengtes (de lengte van de slinger en de beginuitwijking). Voor vervolgonderzoek is aan te raden te zoeken naar een methode om de startpositie nauwkeuriger te bepalen.
	
	Een kanttekening is dat de luchtweerstand in ons model niet is meegenomen. Dit moet in overweging worden genomen bij vervolgonderzoek.
	
	\subsection{Vervolgonderzoek}
	Naar aanleiding van dit onderzoek hebben wij besloten om de hier behandelde slingerbeweging verder uit te diepen. Wij willen zoeken naar een nauwkeurigere meetmethode en wij willen het model benaderen met een functie. Deze twee samen vormen de uitdaging van ons profielwerkstuk.
	
	\appendix
	\clearpage
	
	\begin{thebibliography}{9}
		\addcontentsline{toc}{section}{Referenties}
		
		\bibitem{Lissajous}
		Lissajousfiguur. (z.d.). Geraadpleegd op 30 oktober 2017, van https://nl.wikipedia.org/wiki/Lissajousfiguur
		
		\bibitem{Hooke}
		Wet van Hooke. (z.d.). Geraadpleegd op 30 oktober 2017, van https://nl.wikipedia.org/wiki/Wet\_van\_Hooke
		
		\bibitem{Bruce}
		Yeany, B. (2017, 28 maart). Sand pendulums – Lissajous patterns – part one // Homemade Science with Bruce Yeany [Video]. Geraadpleegd van https://www.youtube.com/watch?v=uPbzhxYTioM
		
	\end{thebibliography}


	\clearpage
	\section{Logboek}
	Periode: oktober 2017 tot juni 2018
	\begin{table}[h]
		\centering

		\begin{tabular}{|l|l|r|r|l} \cline{1-4}
			Datum   & Activiteit				  & \textbf{Vaughan} & \textbf{Jochem} & \\ \cline{1-4}
			26 - 27 okt & Model opstellen      & 480       &           & \\ \cline{1-4}
			27 okt  & Model testen       	   & 210       &           & \\ \cline{1-4}
			30 okt - 7 nov & Opzet onderzoek   &           & 140       & \\ \cline{1-4}
			1 nov   & Opstellen hoofdvraag     & 20        & 20        & \\ \cline{1-4}
			        & Bespreking met dhr. Schuurbiers & 50 & 50        & \\ \cline{1-4}
			        & Model opstellen          & 400       &           & \\ \cline{1-4}
			8 nov   & Opzet verslag            &           & 50        & \\ \cline{1-4}
			8 - 23 nov & Model opstellen       & 700       & 50        & \\ \cline{1-4}
			23 nov	& Model opstellen en testen & 360      &           & \\ \cline{1-4}
			25 nov  & Eerste practicum         & 190       & 190       & \\ \cline{1-4}
			        & Model opstellen          &           & 50        & \\ \cline{1-4}
			8 dec   & Bespreking met dhr. van Saagsvelt & 70 & 70      & \\ \cline{1-4}
			3 feb - 24 mrt & Model opstellen    & 620      &           & \\ \cline{1-4}
			22 mrt  & Bespreking proefopstelling & 40      & 40        & \\ \cline{1-4}
			1 apr   & Model invoeren in Python &           & 200       & \\ \cline{1-4}
			1 - 3 apr & Verslag                &           & 180       & \\ \cline{1-4}
			6 apr   & Proefopstelling bouwen   & 270       & 200       & \\ \cline{1-4}
			7 apr   & Verslag                  &           & 80        & \\ \cline{1-4}
			8 apr   & Proefmetingen            & 240       & 240       & \\ \cline{1-4}
			        & Verslag                  &           & 70        & \\ \cline{1-4}
			13 apr  & Model opstellen          & 150       &           & \\ \cline{1-4}
			14 apr  & Verslag, model en metingen & 540     & 600       & \\ \cline{1-4}
			14 - 22 mei  & Verslag             &           & 170       & \\ \cline{1-4}
			28 - 31 mei & Model opstellen      & 170       &           & \\ \cline{1-4}
			5 - 7 jun & Model uitschrijven     & 1030      & 80        & \\ \cline{1-4}
			8 jun   & Verslag                  & 60        & 150       & \\ \cline{1-4}
			9 jun   & Metingen                 & 510       & 530       & \\ \cline{1-4}
			10 - 11 jun & Verslag              & 150       & 630       & \\ \cline{1-4}
			12 jun  & Verslag                  & 540       & 540       & \\ \hline \hline
			\multicolumn{2}{r|}{Totaal}        & 6790 min  & 4330 min  & = 11120 min \\ \cline{3-4}
			\multicolumn{2}{r|}{}              & 113 uur   & 72 uur    & = 185 uur\\ \cline{3-4}
			\multicolumn{4}{l}{\textbf{Leren voor het PO}} & \\ \cline{1-4}
			jan - feb & Python \& wiskundig modelleren &   & 2400      & \\ \cline{1-4}
			2 apr  & \LaTeX \space zelfstudie &            & 120       & \\ \hline \hline
			\multicolumn{2}{r|}{Totaal}       & 6790 min   & 6850 min  & = 13640 min \\ \cline{3-4}
			\multicolumn{2}{r|}{}             & 113 uur    & 114 uur   & = 227 uur \\ \cline{3-4}
		\end{tabular}
	\end{table}
	
	\clearpage
	\section{Python model}
	\begin{lstlisting}
"""
Created on Sun Apr  1 12:54:36 2018
@author: Jochem van Rabenswaaij
"""

import numpy as np 
import matplotlib.pyplot as plt
import pandas as pd

#Initializations
alpha_init = 0.086635778
beta_init = 0.1542188
AP = 0.97
a = 0.255

alphav_init = 0
betav_init = 0
dt = 0.001
g = 9.81
t_init = 0
t_end = 9

n_steps = int(round((t_end-t_init)/dt)) # total number of timesteps
c3 = -AP - a

t_arr = np.zeros(n_steps + 1)
a_arr = np.zeros((2,n_steps+1))
v_arr = np.zeros((2,n_steps+1))
alpha_arr = np.zeros((2,n_steps+1))
x_arr = np.zeros((2,n_steps+1))

t_arr[0] = t_init
v_arr[0,0] = alphav_init
v_arr[1,0] = betav_init
alpha_arr[0,0] = alpha_init
alpha_arr[1,0] = beta_init

#Euler's method
for i in range (1, n_steps + 1): 
	#Load everything
	t = t_arr[i-1]
	alphav = v_arr[0,i-1]
	betav = v_arr[1,i-1]
	alpha = alpha_arr[0,i-1]
	beta = alpha_arr[1,i-1]
	#Calculate new values
	p1 = AP * np.sin(beta)
	p2 = AP * np.sin(alpha) * np.cos(beta) + a * np.sin(alpha)
	p3 = -AP * np.cos(alpha) * np.cos(beta) - a * np.cos(alpha)
	
	b1 = AP * np.sin(beta)
	b3 = -AP * np.cos(beta) - a
	
	alphaa_new = -g * abs(np.sin(alpha)) * np.sqrt((b3 * p2)**2 
		+ (b3 * p1 - b1 * p3)**2 + (b1 * p2)**2) * (b3 * c3) 
		/ (np.sin(alpha) * (b1**2 + b3**2)**2 * abs(c3))
	betaa_new = -g * np.sin(beta) * np.cos(alpha) / AP
	
	alphav_new = alphav + alphaa_new * dt
	betav_new = betav + betaa_new * dt
	
	alpha_new = alpha + alphav_new *dt
	beta_new = beta + betav_new *dt
	
	#Store new values
	a_arr[0,i] = alphaa_new
	a_arr[1,i] = betaa_new
	v_arr[0,i] = alphav_new
	v_arr[1,i] = betav_new
	alpha_arr[0,i] = alpha_new
	alpha_arr[1,i] = beta_new
	x_arr[0,i-1] = p2
	x_arr[1,i-1] = p1
	t_arr[i] = t + dt  
	
	x_arr[0,i] = p2
	x_arr[1,i] = p1

csv_file = pd.read_csv('r6.csv')

#Plot the results
fig = plt.figure() 
plt.plot(x_arr[1,:], x_arr[0,:], linewidth = 2, label = "model")
plt.plot(csv_file['P1Y'], csv_file['P1X'], linewidth = 2,
	label = "metingen")

plt.xlabel('x [m]', fontsize = 20) # name of horizontal axis 
plt.ylabel('y [m]', fontsize = 20) # name of vertical axis
plt.xticks(fontsize = 15) # adjust the fontsize 
plt.yticks(fontsize = 15) # adjust the fontsize 
plt.axis([-.2, .2, -.2, .2]) # set the range of the axes
plt.legend(fontsize=10) # show the legend 
plt.show() # necessary for some platforms

fig.savefig('Lissajous.jpg', dpi=fig.dpi, bbox_inches = "tight")


fig = plt.figure() 
plt.plot(t_arr[:], x_arr[0,:], linewidth = 2, label = "model")
plt.plot(csv_file['time']-0.4, csv_file['P1X'], linewidth = 2,
	label = "metingen")

plt.xlabel('time [s]', fontsize = 20) # name of horizontal axis 
plt.ylabel('y [m]', fontsize = 20) # name of vertical axis
plt.xticks(fontsize = 15) # adjust the fontsize 
plt.yticks(fontsize = 15) # adjust the fontsize 
plt.axis([0, 5, -0.2, .2]) # set the range of the axes
plt.legend(fontsize=10) # show the legend 
plt.show() # necessary for some platforms

fig.savefig('Lissajous1.jpg', dpi=fig.dpi, bbox_inches = "tight")


fig = plt.figure() 
plt.plot(t_arr[:], x_arr[1,:], linewidth = 2, label = "model")
plt.plot(csv_file['time']-0.4, csv_file['P1Y'], linewidth = 2, 
	label = "metingen")
 
plt.xlabel('time [s]', fontsize = 20) # name of horizontal axis 
plt.ylabel('x [m]', fontsize = 20) # name of vertical axis
plt.xticks(fontsize = 15) # adjust the fontsize 
plt.yticks(fontsize = 15) # adjust the fontsize 
plt.axis([0, 5, -.2, .2]) # set the range of the axes
plt.legend(fontsize=10) # show the legend 
plt.show() # necessary for some platforms

fig.savefig('Lissajous2.jpg', dpi=fig.dpi, bbox_inches = "tight")
	\end{lstlisting}
		
	\clearpage
	\addtocounter{section}{1}
	\addcontentsline{toc}{section}{\protect\numberline{\thesection}Certificaat}
	\includepdf{certificaat.pdf}

	
\end{document}